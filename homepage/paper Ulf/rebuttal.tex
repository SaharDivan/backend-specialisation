We are happy that the referee describes our paper as on "an important fundamental topic", "competently written" and as a paper that "certainly deserves publication". 

However, we disagree that it would only interest the authors of Refs. [30-33] (Refs. [29,30,37-39] in the revised paper). One sees this from the debate of Refs. [12-28] that was the subject of the Nature Forum of Ref. [11]. Our paper is very much relevant to this debate. Note that Nature would not have invited Tyc and Zhang to write on the issue if it were not relevant to a wide readership. Therefore we think that our paper should also find a wider audience. 

On the other hand, our paper is necessarily specialised, as any research paper must be. We are aware of the large body of work on superresolution and agree with the referee that some of it should be mentioned - we do this now in Ref. [4] in the revised paper where we refer to the pioneering papers of Gabor, di Francia and Harris. However, it might be unreasonable to expect that a research paper is the right venue for surveying the entire literature on a field with a long history. We think that one must pick one subject and one angle of attack to the problem, knowing of course that this represents a simplification. We believe that Feynman's remark on the time-invariance of Maxwell's equations makes a good point of entrance, because it is a simple and general argument. We also follow the suggestion of the referee and cite Carminati et al., PRA 62, 012712 (2000) on the issue of reciprocity, unitarity, and time-reversal symmetry in the presence of evanescent waves, as this paper is related to Feynman's argument and to Rosny's and Fink's experiment [36]. 

We agree that one can use computational methods for phase retrieval and superresolution where then noise and uncertainty in the data becomes absolutely crucial. Note that absolute optical instruments such as Maxwell's fish eye perform the reversal of diffraction - not computationally of course, but physically in the devices themselves. The do not seem to be strongly susceptible to noise, because otherwise the experiment of Ref. [30] would not have worked. Noise is always present in experiments. In the revised paper, we referred to Synge's work on absolute optical instruments and to Nieto-Vesperinas's book on Scattering and Diffraction in Physical Optics on the problem of noise and uncertainty in image data. 

We agree with the referee that many well-known results of classical optics have been ignored in the discussion of the perfect lens made by negative refraction, but please note that this is not the subject of our paper and we should not be blamed for the errors of others. We also agree that noise is an issue in our case, but it is not the most fundamental problem to be sorted out first in the case we are studying; the interactions between the sources and drains need to be understood first, because they impose the most fundamental limitations. 

We also agree with the referee that some of the mathematics is heavy-going, especially at the beginning, and that readers should get some reward in studying a simple example. But is not Sec. IIIB such an example? Everything in this subsection is done analytically, the results are intuitive and can be fully understood. Note also that Fig. 6 illustrates the fairly general argument of Sec. IIIC with a concrete example, the case of Fig. 1 our paper starts with. Here the detailed mathematical expressions are more complicated, and we have omitted them for this reason, but readers can easily solve the relevant equations with Mathematica and reproduce the figure. As the referee may have noticed, the figures are connected. They tell the story of the paper in a visual way. To better help the reader with understanding the material, we have included one more figure in the paper (the new Fig. 3). We hope that the figures brighten up the mathematics and encourage readers to plough through the paper.

Finally, we mentioned the incoming and outgoing waves in Sec. IIIB and the fact that we consider coherent sources (the limitations due to interactions only matter for coherent sources).

We hope that these explanations and revisions are satisfactory and that the paper can be accepted. 
