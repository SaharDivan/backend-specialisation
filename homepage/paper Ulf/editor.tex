Dear Editors,

Perfect imaging with positive refraction has been subject to an extensive debate [see T. Tyc and X. Zhang, Nature 480, 42 (2011)] that has focused on the role of detection - a subwavelength image only appears when it is detected, and sources and detectors are strongly interacting with each other. Modelling such interacting sources and drains has been a formidable challenge, full computer simulations have been difficult due to the vastly different scales involved. 

Our manuscript is the first paper that develops a theoretical model of mutually interacting sources and drains in absolute optical instruments. Their interaction turns out to ruin some of the promising features of perfect imaging, but also to open interesting new prospects in scanning near fields from far-field distances.

We contacted Manolis Antonoyiannakis in an informal enquiry, because we were not sure whether PRA or PRB would be the most appropriate journal. Manolis recommended to submit the paper to PRA. 

With best regards,

Ulf Leonhardt and Sahar Sahebdivan

* CT Chan, Hong Kong University of Science and Technology, Hong Kong
* Mathias Fink, Ecole superieure de physique et de chimie industrielles de la ville de Paris, France
* Gerd Leuchs, Max Planck Institute for the Science of Light, Germany
* Juan Carlos Minano, Universidad Politecnica de Madrid, Spain
* Tomas Tyc, Masaryk University, Czech Republic